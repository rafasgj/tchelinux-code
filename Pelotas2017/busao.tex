\documentclass[aspectratio=169,14pt]{beamer}
\usepackage[utf8]{inputenc}
\usepackage[brazil]{babel}
\usepackage{graphics}
\usepackage{transparent}
% Adiciona tux gauderio a todos os slides
\usebackgroundtemplate{
    \includegraphics[width=2cm]{../images/Tuxgaucho.png}
}
% Configura a apresentação para ser executada em tela cheia.
\hypersetup{pdfpagemode=FullScreen}
% Hide beamer navigation simbols
\beamertemplatenavigationsymbolsempty
% Centraliza os títulos dos slides.
\setbeamertemplate{frametitle}[default][center]

%
% Modify from here!
%

\title{Com que ônibus eu vou? \\ \large{Uma introdução à programação com Python}}
\author{Rafael Guterres Jeffman}
\institute{Faculdade Senac Porto Alegre \\ Tchelinux}
\date{9 de Junho de 2017}

\begin{document}

\begin{frame}
    \titlepage
\end{frame}

\begin{frame}
    \frametitle{Aprendendo Linguagens de Programação}
    \begin{itemize}
        \item Sintaxe
        \item Paradigma
        \item Biblioteca
        \item Frameworks
    \end{itemize}
\end{frame}

\begin{frame}
    \begin{center}
    Não adianta ler sobre a linguagem  se você não tiver um problema
    para resolver...
    \end{center}
\end{frame}

\begin{frame}
    \frametitle{O Problema}
    
    \vfill
    \begin{center}
        Estou em Porto Alegre, e preciso ir a um lugar, mas não faço
        idéia de qual ônibus devo utilizar...
    \end{center}
    \vfill
    \vfill
    \hfill
    \tiny{Infelizmente, não achei os dados para a cidade de Pelotas...}
\end{frame}

\begin{frame}
    \frametitle{A solução}
    \begin{itemize}
        \setlength\itemsep{1em}
        \item \textbf{GTFS} General Transit Feed Specification
        \item Árvores K-D
        \item Tabelas de Espalhamento
        \item Listas
    \end{itemize}
\end{frame}

\begin{frame}
    \frametitle{O Algoritmo}
    \vfill
    \small \begin{enumerate}
        \item Obtém as coordenadas GPS da origem e destino.
        \item Obtém as paradas mais próximas do destino.
        \item Obtém as paradas mais próximas da origem.
        \item Para cada parada próxima do destino, ordenada por
        proximidade, compara a lista de ônibus que atende a parada com
        a lista de ônibus que atendem as paradas próximas à origem.
        \item O primeiro ônibus encontrado, é a resposta.
    \end{enumerate}
\end{frame}

\begin{frame}
    \frametitle{A linguagem}
    \begin{center}
        \includegraphics[height=0.5\paperheight]{../images/pythonlogo.jpg}
    \end{center}
\end{frame}

\begin{frame}
    \frametitle{Por que Python?}
    \vfill
    \begin{itemize}
        \item Sintaxe elegante que facilita a leitura
        \item Suporte a diversos paradigmas (orientação a objetos,
        estruturado, funcional)
        \item Eficiente para prototipação rápida
        \item Se necessário, pode ser facilmente extendida com módulos
        em outras linguagens
    \end{itemize}
\end{frame}

\begin{frame}
    \vfill
    \begin{center}
        \huge Vamos para o código?
    \end{center}
\end{frame}

{
\usebackgroundtemplate{
    {
        \includegraphics[width=1.05\paperheight]{../images/Tuxgaucho.png}
    }
}
\begin{frame}
    \begin{flushright}
    \huge \textbf{Thank you!} \\
    \vfill
    \small \textbf{rafasgj@gmail.com}
    \end{flushright}
\end{frame}
}

\end{document}
