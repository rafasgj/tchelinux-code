\documentclass[aspectratio=169,14pt]{beamer}
\usepackage[utf8]{inputenc}
\usepackage[brazil]{babel}
\usepackage{graphics}
\usepackage{tikz}
\usepackage{transparent}
% Adiciona tux gauderio a todos os slides
\usebackgroundtemplate{
    {
        \transparent{0.2}
        \includegraphics[width=1.1\paperheight]{../../images/Tuxgaucho.png}
    }
}
% Configura a apresentação para ser executada em tela cheia.
\hypersetup{pdfpagemode=FullScreen}
% Hide beamer navigation simbols
\beamertemplatenavigationsymbolsempty
% Centraliza os títulos dos slides.
\setbeamertemplate{frametitle}[default][center]

%
% Modify from here!
%
\setbeamercolor{frametitle}{fg=black}
\setbeamerfont{frametitle}{series=\bfseries}

\usepackage{multicol}

\title{Distribuições Linux}
\author{Rafael Guterres Jeffman}
\institute{Faculdade Senac Porto Alegre \\ Tchelinux}
\date{}

\begin{document}

\begin{frame}
    \titlepage
\end{frame}

\begin{frame}
    \frametitle{Pré-História}
    \begin{description}
        \item[1970]{É lançado o UNIX (AT\&T).}
        \item[1977]{É lançado o BSD.}
        \item[1983]{Richard Stallman inicia o projeto GNU.}
        \item[1985]{É lançado o Intel 80386.}
        \item[1987]{É lançado o Minix.}
    \end{description}
\end{frame}

\begin{frame}
    \frametitle{The Beginning}
    From: torvalds@klaava.Helsinki.FI (Linus Benedict Torvalds) \\
    Newsgroups: comp.os.minix \\
    Subject: Gcc-1.40 and a posix-question \\
    Message-ID: Date: 3 Jul 91 10:00:50 GMT \\
    \vspace{1em}
    Hello netlanders, \\
    Due to a project I'm working on (in minix), I'm interested in the
    posix standard definition. Could somebody please point me to a
    (preferably) machine-readable format of the latest posix rules? \\
    Ftp-sites would be nice.
\end{frame}

\begin{frame}
    \frametitle{A História}
    \framesubtitle{Anos 90 - O Início}
    \begin{description}
   \item[1991]{O kernel do Linux é anunciado em 25 de Agosto.}
   \item[1992]{A licença do kernel muda para GNU GPL. Aparecem as distribuições.}
   \item[1993]{É criada a Slackware, hoje, a distribuição mais longeva do Linux.}
   \item[1994]{Versão 1.0 do kernel.}
   \item[1995]{Linux funciona no DEC Alpha e na Sun SPARC.}
   \item[1996]{Versão 2.0 do kernel com suporte a SMP.}
   \item[1998]{Grandes empresas começam a suportar o Linux (IBM, Compaq e Oracle).}
    \end{description}
\end{frame}

\begin{frame}
    \frametitle{A História}
    \framesubtitle{Anos 2000 - O crescimento.}
    Durante este período, o \emph{Kernel} do Linux teve um crescimento em
    relação ao hardware suportado, e a sua facilidade de uso, no entanto,
    a grande diferença ocorreu com relação aos aplicativos.
    \vspace{1em}
    \begin{description}
        \item[2000]{A Dell anuncia a disponibilidade do Linux em toda sua linha.}
        \item[2007]{A Dell começa a distribuir notebooks com o Ubuntu.}
        \item[2009]{2009: A Red-Hat, empresa cujos produtos são software
        livre, tem o mesmo valor de mercado que a Sun (criadora do Java).}
    \end{description}
\end{frame}

\begin{frame}
    \frametitle{A História}
    \framesubtitle{Anos 2010 - O Domínio do Mundo.}
    \begin{description}
        \item[2011]{Versão 3.0 do kernel é lançada.}
        \item[2012]{O faturamento do mercado de Linux ultrapassa o
        faturamento do mercado do UNIX.}
        \item[2013]{A Google diz que 75\% do mercado de \emph{smartphones}
        utiliza Android, baseado no \emph{kernel} do Linux.}
        \item[2014]{A Canonical diz que o Ubuntu possui 22 Milhões de
        usuários.}
        \item[2015]{A versão 4.0 do \emph{kernel} do Linux é lançada.}
        \item[2016]{Na lista de novembro, com exceção de
        dois computadores, \textbf{todos} da lista dos 500 mais potentes
        do mundo, executam o \textit{kenel} do Linux.}
    \end{description}
\end{frame}

\begin{frame}
    \frametitle{O Kernel}
    \begin{itemize}
        \item O que faz um kernel?
        \item Por que eu preciso de um kernel?
        \item Por que não adianta só ter um kernel?
    \end{itemize}
\end{frame}

{%
\setbeamercolor{background canvas}{bg=white}
\setbeamercolor{normal text}{fg=white}
\begin{frame}
    \frametitle{Distribuições Linux}
    \begin{itemize}
        \item{O que são?}
        \item{Onde vivem?}
        \item{Do que se alimentam?}
    \end{itemize}
\end{frame}
}

\begin{frame}
    \frametitle{Sistema Operacional \\ não é só Kernel}
    \begin{itemize}
        \item Ferramentas de gerenciamento.
        \item Ferramentas de iteração com usuário.
        \item Aplicações utilizam serviços do kernel para prestar
        serviços aos usuários.
    \end{itemize}
\end{frame}

\begin{frame}
    \frametitle{O que é uma e o que faz \\ uma Distribuição Linux?}
    \begin{itemize}
        \item Distribuição Linux é uma coleção de softwares.
        \item Gerenciamento do Sistema.
        \item Gerenciamento de Usuários.
        \item Ferramentas básicas.
        \item Aplicações.
        serviços aos usuários.
    \end{itemize}
\end{frame}

\begin{frame}
    \begin{center}
    \Huge \textbf{A História \\ das Distribuições Linux}
    \end{center}
\end{frame}

\begin{frame}
    \frametitle{1992}
    \begin{description}
        \item[MCC] Primeira distribuição Linux.
        \item [SLS] Segunda distribuição Linux. Talvez a que tenha tido
        mais problemas, mas, na época, a que tinha mais potencial.
    \end{description}
\end{frame}

\begin{frame}
    \frametitle{1993}
    \begin{center}\large Slackware \end{center}
    \begin{itemize}
        \item Patrick Volkerding, desenvolvedor que utilizava
        a SLS criou o Slackware
        \item O objetivo era tentar resolver os problemas do SLS.
        \item Ele conseguiu.
        \item É a distribuição mais longeva do Linux.
        \item Última \emph{Release}: Julho de 2016.
    \end{itemize}
\end{frame}

\begin{frame}
    \frametitle{1993}
    \begin{center}\large Debian \end{center}
    \begin{itemize}
      \item Também foi criado devido a problemas com o SLS.
      \item Ian Murdoch, criou uma distribuição onde todos programas
      deveriam ser 100\% livres.
      \item É a distribuição que mais influenciou a criação de novas distribuições.
    \end{itemize}
\end{frame}

\begin{frame}
    \frametitle{1994}
    \begin{center}\large SuSE \end{center}
    \begin{itemize}
        \item Originalmente baseado no Slackware, surgiu na Alemanha.
    \end{itemize}
    \begin{center}\large Red Hat \end{center}
    \begin{itemize}
        \item Criada nos EUA, por um grupo de amigos.
        \item Transformou-se na maior empresa de software livre do mundo.
    \end{itemize}
\end{frame}

\begin{frame}
    \frametitle{1997}
    \begin{center}\large Linux PPC \end{center}
    \begin{itemize}
        \item Primeira distribuição Linux para os processadores RISC PowerPC.
    \end{itemize}
    \begin{center}\large Conectiva\end{center}
    \begin{itemize}
        \item Principal distribuição brasileira.
        \item Da empresa saíram diversos desenvolvedores proeminentes no
        mundo Linux, como Marcelo Tossatti, mantenedor do kernel 2.4.
    \end{itemize}
\end{frame}

\begin{frame}
    \frametitle{1998}
    \begin{description}
        \item[$\mu$CLinux] Linux baseado na $\mu$LibC para dispositivos embarcados.
        \item[Corel] Mais uma tentativa de salvar a empresa, agora com uma distribuição Linux.
        \item[Mandrake] Trouxe o Plug N' Play (de verdade) para o Linux.
    \end{description}
\end{frame}

\begin{frame}
    \frametitle{1999 / 2000}
    \begin{center}\large Linux From Scratch \end{center}
    Na virada do século, surge o Linux From Scratch, um conjunto de documentos,
    \emph{patches} e \emph{scripts} que auxilia na criação de novas distribuições
    Linux.
\end{frame}

\begin{frame}
    \frametitle{2000}
    \begin{description}
        \item[SLES] A SuSE lança a versão \emph{enterprise} da sua distribuição.
        \item[RHAS] A Red Hat lança a versão \emph{enterprise} da sua
        distribuição, a Red Hat Advanced Server.
        \item[Knoppix] A distribuição argentina mais conhecida, trazia um
        excelente sistema de detecção e configuração de \emph{hardware Plug-and-Play}.
    \end{description}
\end{frame}

\begin{frame}
    \frametitle{2001}
    \begin{description}
        \item[OpenWRT] Distribuição que foi muito utilizada em roteadores
        WiFi (Linksys).
    \end{description}
\end{frame}

\begin{frame}
    \frametitle{2002}
    \begin{description}
        \item[Arch] O Arch Linux é uma das distribuições mais configuráveis
        e uma excelente fonte de documentação sobre programas e configurações
        necessárias ao funcionamento do Linux.
        \item[Gentoo] Rovolucionou a distribuição de pacotes a partir do
        código fonte.
        \item[GoboLinux] Distribuição gaúcha que apresenta uma árvore de
        diretórios completamente diferente, e modifica a forma como os
        pacotes são gerenciados.
    \end{description}
\end{frame}

\begin{frame}
    \frametitle{2003}
    \begin{description}
        \item[RHEL] A Red Hat muda o nome da sua versão \emph{Enterprise},
        e a direciona para toda a linha de equipamentos de uma companhia
        (servidores e workstations).
        \item[DSL] O Damn Small Linux era ma distribuição que cabia em
        um único disquete. Lançou a idéia de pequenas distribuições
        focadas em uma única tarefa.
        \item[Fedora] A Red Hat separa a distribuição para usuários de
        desktop do produto principal.
    \end{description}
\end{frame}

\begin{frame}
    \frametitle{2004}
    \begin{description}
        \item[CentOS] Lançado por uma comunidade, é uma versão livre do
        Red Hat Enterprise Linux, e tem apoio da empresa.
        \item[Ubuntu] Em outubro, é lançada pela Cannonical a primeira versão do Ubuntu,
        com um sistema de distribuição e marketing que a fez rapidamente
        virar a distribuição Linux mais utilizada no mundo, além de fazer
        com que mais pessoas tivessem contato com o sistema.
    \end{description}
\end{frame}

\begin{frame}
    \frametitle{2005}
    \begin{description}
        \item[DNA] Baseado no Slackware, o DNA Linux é uma distribuição
        focada nas pesquisas com Genoma Humano.
    \end{description}
\end{frame}

\begin{frame}
    \frametitle{Distribuições mais influentes}
    \begin{itemize}
        \item Debian
        \item Red Hat
        \item Slackware
    \end{itemize}
\end{frame}

\begin{frame}
    \begin{center}
    \huge \textbf{http://futurist.se/gldt/}
    \end{center}
\end{frame}

\begin{frame}
    \frametitle{Por que tantas distribuições?}
    \begin{itemize}
        \item Problemas diferentes.
        \item Programas com versões diferentes.
        \item Visão diferente de como a comunidade contribuirá.
        \item Diferentes formas de gerenciar a distribuição.
        \item Modelos de Negócio diferentes.
    \end{itemize}
\end{frame}

\begin{frame}
    \frametitle{Criando uma distribuição Linux}
    \begin{itemize}
        \item Linux From Scratch
        \item Buildroot
        \item OpenEmbedded
        \item Yocto
    \end{itemize}
\end{frame}

\begin{frame}
    \frametitle{E uma micro-mini-nano distribuição?}
    \begin{itemize}
        \item Kernel
        \item Módulos de Hardware
        \item Inicialização do Sistema
        \item Busybox - o canivete suíço.
    \end{itemize}
\end{frame}

\begin{frame}
    \begin{center}
    \huge \textbf{Qual é a melhor distribuição Linux?}
    \end{center}
\end{frame}

{
\setbeamertemplate{background canvas} [vertical shading][top=cyan,bottom=violet]
%\usebackgroundtemplate[default]
\begin{frame}
    \includegraphics[width=0.9\paperwidth]{../../images/gobolinux.png}
\end{frame}
}

\begin{frame}
    \begin{center}
    \huge \textbf{Por quê?}
    \end{center}
\end{frame}

\begin{frame}
    \begin{center}
    \huge \textbf{Por quê eu ajudei a desenvolver!}

    \huge \textbf{;-)}
    \end{center}
\end{frame}

{
\usebackgroundtemplate{
    {
        \transparent{1.0}
        \includegraphics[width=1.1\paperheight]{../../images/Tuxgaucho.png}
    }
}
\begin{frame}
    \begin{flushright}
    \huge \textbf{Thank you!}
    \end{flushright}
\end{frame}
}

\begin{frame}
    \frametitle{Links Úteis}
    http://kernel.org

    http://distrowatch.com

    http://futurist.se/gldt/

    http://tchelinux.org

    http://gobolinux.org
\end{frame}

\end{document}
