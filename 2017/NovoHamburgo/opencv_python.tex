\documentclass[aspectratio=169,14pt]{beamer}
\usepackage[utf8]{inputenc}
\usepackage[brazil]{babel}
\usepackage{graphics}
\usepackage{tikz}
\usepackage{transparent}
% Adiciona tux gauderio a todos os slides
\usebackgroundtemplate{
    {
        \transparent{0.2}
        \includegraphics[width=1.1\paperheight]{../../images/Tuxgaucho.png}
    }
}
% Configura a apresentação para ser executada em tela cheia.
\hypersetup{pdfpagemode=FullScreen}
% Hide beamer navigation simbols
\beamertemplatenavigationsymbolsempty
% Centraliza os títulos dos slides.
\setbeamertemplate{frametitle}[default][center]

\setbeamercolor{frametitle}{fg=black}
\setbeamerfont{frametitle}{series=\bfseries}

%
% Modify from here!
%

\usepackage{ragged2e}

\title[]{\color{black} \textbf{Phyton e OpenCV \\ Uma introdução prática.}}
\author[]{Rafael Guterres Jeffman}
\institute[]{Faculdade Senac Porto Alegre \\ Tchelinux}
\date{Agosto de 2017}

\begin{document}

\begin{frame}
    \titlepage
\end{frame}


\begin{frame}
    \frametitle{Python}
    \begin{itemize}
        \item{Desenvolvida com o objetivo de facilitar a programação para
        físicos e matemáticos.}
        \item{Multi-paradigma, dinâmica, multi-plataforma.}
        \item{Facilidade de integração com C.}
        \item{Focada na legibilidade.}
    \end{itemize}
\end{frame}

\begin{frame}
    \frametitle{OpenCV}
    \begin{itemize}
        \item{Biblioteca \emph{open source} dedicada à visão computacional
        e \emph{machine learning}.}
        \item{Possui mais do 2500 algoritmos otimizados.}
        \item{Abstrai o acesso ao hardware, como câmeras e GPU.}
        \item{Provê uma abstração simples para criação de interfaces
        homem-máquina.}
        \item{Multi-plataforma: Linux, Mac, Windows, iOS e Android.}
        \item{Suporte a diferentes linguagens de programação, como
        C, C++, Objective-C, Python e Java.}
    \end{itemize}
\end{frame}

\begin{frame}
    \frametitle{OpenCV e Python}
    \begin{itemize}
        \item{Fácil de usar.}
        \item{Rápido de prototipar.}
        \item{Portável e Eficiente.}
    \end{itemize}
\end{frame}

\begin{frame}[c]
    \begin{center}
    \Huge \textbf{Demonstração}
    \end{center}
\end{frame}

\begin{frame}[b]
    \begin{tikzpicture}[remember picture,overlay]
        % Posiciona o overlay na página
        \node [xshift=0.5\paperwidth, yshift=-0.5\paperheight] at (current page.north west)
        % Adiciona a imagem, definindo seu tamanho
        {\includegraphics[width=\paperwidth]{../../images/thats-all-folks.jpg}};
    \end{tikzpicture}

    \begin{flushright}
    \color{white} \textbf{rafasgj@gmail.com}
    \end{flushright}

\end{frame}

\begin{frame}
    \frametitle{Links}
    \begin{itemize}
    \item{\textbf{Python} \\ $\:$ https://python.org}
    \item{\textbf{OpenCV} \\ $\:$ https://opencv.org}
    \item{\textbf{Python Online Course} \\ $\:$ https://codeacademy.com/learn/python}
    \item{\textbf{OpenCV Tutorial} \\ $\:$ https://docs.opencv.org/doc/tutorials/tutorials.html}
    \item{\textbf{OpenCV e Python} \\ $\:$ https://learnopencv.com}
    \end{itemize}
\end{frame}

\end{document}
