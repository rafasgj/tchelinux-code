\documentclass[aspectratio=169,14pt,usenames,dvipsnames]{beamer}

\usepackage[utf8]{inputenc}
\usepackage{enumitem}
\usepackage{calc}

\usepackage{datetime}
\newcommand\builddate{%
   \ifcase \month%
        \or Janeiro%
        \or Fevereiro%
        \or Março%
        \or Abril%
        \or Maio%
        \or Junho%
        \or Julho%
        \or Agosto%
        \or Setembro%
        \or Outubro%
        \or Novembro%
        \or Dezembro%
    \fi\space\number\year%
}

\newcommand{\loadtheme}[1]{%
    \input{themes/#1}%
}
\newcommand{\presentationlanguage}[1]{%
    \usepackage[#1]{babel}%
}

\newcommand{\usecodingsamples}[1]{%
    \usepackage{listings}%
    \input{listings/#1}%
}

% Configura a apresentação para ser executada em tela cheia.
\newcommand{\setfullscreen}{\hypersetup{pdfpagemode=FullScreen}}

% Hide beamer navigation simbols
\beamertemplatenavigationsymbolsempty

%
% Standard frames
%

% coverframe
\newcommand{\coverframe}{%
    \begin{frame} %
        \titlepage %
    \end{frame} %
}

% finalframe{email}
\newcommand{\finalframe}[2][Thank you!]{%
    \begin{frame}%
        \begin{flushright}%
            \huge \textbf{#1}%
            \vfill%
            \large \textbf{#2}%
        \end{flushright}%
    \end{frame}%
}

\newcommand{\bigtitle}[1]{%
    \begin{frame}%
        \begin{center}%
            \Huge {#1}%
        \end{center}%
    \end{frame}%
}



\loadtheme{tchelinux}
\usecodingsamples{python}

\usepackage{setspace}

\title{Não deixe para testar depois o que você pode testar antes.}
\author{Rafael Guterres Jeffman}
\institute{Tchelinux}
\date{2018}

\newcommand\bddfeature[2]{{\color{red}Feature: }{#1\\}{\hspace{1cm}#2\\}}
\newcommand\bddscenario[1]{{\color{blue!75!black}Scenario: }{#1\\}}
\newcommand\bddgiven[1]{{\color{green!50!black}Given }{#1\\}}
\newcommand\bddwhen[1]{{\color{green!50!black}When }{#1\\}}
\newcommand\bddthen[1]{{\color{green!50!black}Then }{#1\\}}
\newcommand\bddand[1]{{\color{green!50!black}And }{#1\\}}
\newcommand\bddbut[1]{{\color{green!50!black}But }{#1\\}}

\makeatletter
\preto{\@verbatim}{\topsep=0pt \partopsep=0pt}
\makeatother

\begin{document}

\coverframe

% 01
\bigtitle{Por que desenvolvemos software?}

% 02
\bigimage{images/bubbaj.jpg}

% 03
\bigtitle{Como desenvolvemos software?}

% 04
\bigimage{images/walter.png}

% 05
\begin{frame}
    \frametitle{Ciclo do Desenvolvimento de Software}
    \begin{itemize}
        \item Análise
        \item Projeto
        \item Implementação
        \item Testes
        \item Implantação
    \end{itemize}
\end{frame}

% 06
\begin{frame}
    \frametitle{User Stories}
    \begin{itemize}
        \item Descrição informal de uma funcionalidade.
        \item Escritas na perspectiva do usuário.
        \item As histórias devem dizer \textbf{quem} é o usuário,
        \textbf{o que} ele vai fazer, e qual o \textbf{objetivo}
        ele quer atingir.
    \end{itemize}
\end{frame}

% 07
\begin{frame}
    \frametitle{Critérios de Aceitação}
    Para determinar se uma história de usuário está completa,
    deve haver um \textbf{critério de aceitação}, com objetivos
    que podem ser medidos e avaliados.
\end{frame}

% 08
\begin{frame}
    \frametitle{Test Driven Development}
    \begin{itemize}
        \item Faz com que todo código escrito seja testado.
        \item Auxilia no design da aplicação.
        \item Inverte a ordem tradicional de testes de software.
    \end{itemize}
    \setstretch{1.5}\color{blue}
    \hspace{1cm}Cria teste $\rightarrow$ Teste falha $\rightarrow$\\
    \hspace{3cm}Implementa código $\rightarrow$ Teste passa $\rightarrow$\\
    \hspace{6cm}Refatorar código
    
\end{frame}

% 09
\begin{frame}
    \frametitle{Críticas ao TDD}
    \begin{itemize}[label=]
        \item Definição do que é uma \textbf{unidade}
        \item Quantidade de código de teste.
        \item Uso de \textbf{mocks}.
        \item Dificuldade de testar algumas situações, com bancos de dados
        e configurações específicas de plataforma.
    \end{itemize}
\end{frame}

% 10
\begin{frame}
    \frametitle{Behavior Driven Development}
    \begin{itemize}
        \item A unidade testada, é um requisito.
        \item Testa um comportamento esperado do sistema.
        \item Utiliza uma linguagem específica de domínio.
        \item Faz uso extensivo de ferramentas.
    \end{itemize}
\end{frame}

% 11
\begin{frame}
    \frametitle{Gherkin DSL}
    \begin{itemize}
        \item Desenvolvida para definir casos de teste no Cucumber.
        \item Procura forçar requisitos e resultados claros.
        \item Facilmente traduz os critérios de aceitação de histórias.
        \item Internacionalizável.
    \end{itemize}
\end{frame}

% 12
\begin{frame}
    \frametitle{Exemplo}
    \bddfeature{I want to code.}{
        As a Developer, I want to code, so that I can pay my bills.}
    \bddscenario{I have to eat.}
    \bddgiven{I know how to code.}
    \bddthen{I'm hired to code.}
    \bddthen{I'm payed in pizza.}
\end{frame}

% 13
\begin{frame}
    \frametitle{Behave}
    \begin{itemize}
        \item Behave é uma ferramenta que auxilia no uso de BDD com Python.
        \item Auxilia na automação de testes de aceitação (e unitários).
        \item Facil de integrar ao projeto.
        \item Busca facilitar a criação do código de teste.
    \end{itemize}
\end{frame}

% 14
\begin{frame}
    \frametitle{Features}
    \begin{itemize}
        \item Features são escritas pelos Stakeholders, Analistas, etc.
        \item Uma feature terá, provavelmente, vários cenários.
        \item Podemos utilizar uma História de Usuário como uma feature,
        e os cenários como os testes de aceitação dessa feature.
        \item Descritas em linguagem natural, baseado no formato Gherkin.
    \end{itemize}
\end{frame}

% 15
\begin{frame}
    \frametitle{Cenários}
    \begin{itemize}
        \item Descrevem os passos para validar um cenário de uma feature.
        \item Cada cenário descreve apenas uma característica da feature.
        \item Cenários são criados a partir do ponto de vista do usuário.
        \item Podemos ter cenários para situações de erro.
    \end{itemize}
\end{frame}

% 16
\begin{frame}
    \frametitle{Given, When, Then}
    \vfill
    Cenários são descritos em passos que, se executados corretamente, mostram que
    o cenário está corretamente implementado.
    \vfill
    \begin{description}
        \item[Given]{Utilizado para colocar o sistema em um estado conhecido.}
        \item[When]{Ações são executadas, modificando, ou não, o estado do sistema.}
        \item[Then]{Saídas observadas e avaliadas.}
    \end{description}
\end{frame}

% 17
\begin{frame}[fragile]
    \frametitle{Um cenário completo.}

    \bddscenario{Change data of an inexistent institution.}
    \bddgiven{the database has data for two institutions}
    \bddwhen{changing the data of an institution with id "inexistent" to}
    {\color{magenta}
    \begin{verbatim}
        """
        { "long_name": "A Nice Place to Be" }
        """\end{verbatim}}
    \bddthen{the status code is 404}
\end{frame}

% 18
\begin{frame}
    \frametitle{Testando com o Behave.}
    \begin{itemize}
        \item As \textbf{features} devem ficar em um diretório \textit{features}
        \item As implementações dos \textbf{passos} devem ficar em um subdiretório
        \textbf{steps}, dentro de \textit{features}.
        \item Você pode controlar diversos aspectos dos testes através de 
        parâmetros da linha de comando.
    \end{itemize}
\end{frame}

% 19
{%
\usebackgroundtemplate{}%
\begin{frame}[fragile]
\color{orange}
\begin{verbatim}
You can implement step definitions for undefined steps
with these snippets:

@given(u'the database has data for two institutions')
def step_impl(context):
    raise NotImplementedError(u'STEP: Given the ...')

@when(u'changing the data of an institution...')
def step_impl(context):
    raise NotImplementedError(u'STEP: When changing...')

@then(u'the status code is 404')
def step_impl(context):
    raise NotImplementedError(u'STEP: Then ...')            
\end{verbatim}
\end{frame}
}

% 20
{%
\usebackgroundtemplate{}%
\begin{frame}[fragile]
    \frametitle{Implementando os Testes: \textit{Given}}
    \begin{python}
@given('the database has data for two institutions')
def given_two_institutions_in_two_cities(context):
    """Add two institutions to the database."""
    url = "http://127.0.0.1:5000/api/city"
    for v in context.cidades.values():
        response = requests.post(url, json=v)
        assert response.status_code == 201
    url = "http://127.0.0.1:5000/api/institution"
    for institution in context.institutions:
        response = requests.post(url, json=institution)
        assert response.status_code == 201
    \end{python}
\end{frame}
}

% 21
{%
\usebackgroundtemplate{}%
\begin{frame}[fragile]
    \frametitle{Implementando os Testes: \textit{When}}
    \begin{python}
@when('changing the data of an institution "{inst_id}"')
def when_changing_institution_data(context, inst_id):
    """Request changes to the institution data."""
    context.institution_id = institution_id
    context.input = json.loads(context.text)
    url = "http://127.0.0.1:5000/api/institution/{}"
    url = url.format(institution_id)
    context.response = requests.put(url,
                                    json=context.input)
\end{python}
\end{frame}
}

% 22
{%
\usebackgroundtemplate{}%
\begin{frame}[fragile]
    \frametitle{Implementando os Testes: \textit{Then}}
    \begin{python}
@then('the status code is {status_code}')
def then_verify_status_code(context, status_code):
    """Test the context response status code."""
    expected = int(status_code)
    observed = context.response.status_code
    assert expected == observed 
    \end{python}
\end{frame}
}

% 23
\bigtitle{Implementando o Código\vfill
    {\Large{Agora devemos passar nos testes...}}\vfill
    Começa a diversão!}

% 24
\begin{frame}
    \frametitle{Para onde ir agora?}
    \begin{itemize}
        \item \url{http://github.com/behave}
        \item {\small\url{http://behave.readthedocs.io/en/latest/tutorial.html}}
        \item \url{http://behave.readthedocs.io/en/latest}
    \end{itemize}
\end{frame}

% 25
\finalframe[Obrigado!]{
        {\fontsize{0.34cm}{0cm}\selectfont\url{https://rafaeljeffman.com/tchelinux}}\\
    \small\url{rafasgj@gmail.com}
}

\end{document}
