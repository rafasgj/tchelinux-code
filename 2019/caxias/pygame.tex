\documentclass[aspectratio=169,14pt,usenames,dvipsnames]{beamer}

\usepackage[utf8]{inputenc}
\usepackage{fontspec}
\usepackage{enumitem}
\usepackage{calc}

\usepackage{datetime}
\newcommand\builddate{%
   \ifcase \month%
        \or Janeiro%
        \or Fevereiro%
        \or Março%
        \or Abril%
        \or Maio%
        \or Junho%
        \or Julho%
        \or Agosto%
        \or Setembro%
        \or Outubro%
        \or Novembro%
        \or Dezembro%
    \fi\space\number\year%
}

\newcommand{\loadtheme}[1]{%
    \input{themes/#1}%
}
\newcommand{\presentationlanguage}[1]{%
    \usepackage[#1]{babel}%
}

\newcommand{\usecodingsamples}[1]{%
    \usepackage{listings}%
    \input{listings/#1}%
}

% Configura a apresentação para ser executada em tela cheia.
\newcommand{\setfullscreen}{\hypersetup{pdfpagemode=FullScreen}}

% Hide beamer navigation simbols
\beamertemplatenavigationsymbolsempty

%
% Standard frames
%

% coverframe
\newcommand{\coverframe}{%
    \begin{frame} %
        \titlepage %
    \end{frame} %
}

% finalframe{email}
\newcommand{\finalframe}[2][Thank you!]{%
    \begin{frame}%
        \begin{flushright}%
            \huge \textbf{#1}%
            \vfill%
            \large \textbf{#2}%
        \end{flushright}%
    \end{frame}%
}

% bigtitle{title}
\newcommand{\bigtitle}[1]{%
    \begin{frame}%
        \begin{center}%
            \Huge {#1}%
        \end{center}%
    \end{frame}%
}

% bigimage{file}
\newcommand{\bigimage}[2][1.0]{%
    {%
        \usebackgroundtemplate{}%
        \begin{frame}%
            {%
            \makebox[\textwidth][c]{%
              \includegraphics[height=#1\paperheight, width=#1\paperwidth,%
                               keepaspectratio]{#2}%
              }%
            }%
        \end{frame}%
    }%
}


\usecodingsamples{python}

%\loadtheme{apple_keynote_black}
\loadtheme{invaders}

\title{Desenvolvendo Jogos com pygame}
\subtitle{}
\author{Rafael Guterres Jeffman}
\institute{}
\date{2019}

\begin{document}

%01
\coverframe

%02
\bigtitle{Por que \\ jogos?}

%02
\begin{frame}
    \frametitle{Desenvolver Jogos}

    \begin{columns}
        \begin{column}{0.5\textwidth}
            \begin{itemize}
                \item É divertido.
                \item Tu sempre quis fazer.
                \item Foi a primeira coisa que tu fez com algo que parecia
                um computador.
            \end{itemize}
        \end{column}
        \begin{column}{0.5\textwidth}
            \begin{itemize}
                \item Não precisa ser difícil.
                \item Não é fácil.
                \item Tu quer mostrar pra todo mundo que tu consegue
                desenvolver um jogo.
            \end{itemize}
        \end{column}
    \end{columns}
    \begin{center}
        \item {\Large \textbf{É muito divertido!}}
    \end{center}
\end{frame}

%03
\bigtitle{Por que \\ Python?}

%03.a
\begin{frame}
    \frametitle{Por que Python?}

    \begin{itemize}
        \item É divertido.
        \item Permite que a preocupação seja o problema.
        \item Faz com que tu aprenda uma linguagem que está sendo muito
        utilizada.
    \end{itemize}
\end{frame}

%04
\bigtitle{Por que \\ pygame?}

%04.a
\begin{frame}
    \frametitle{pygame}

    \begin{itemize}
        \item É multi-plataforma (SDL).
        \item É uma biblioteca de componentes.
        \item Retira \textit{as parada chata} da programação de jogos.
        \item Não é um engine de jogos, afinal, queremos programar.
    \end{itemize}
\end{frame}

%05
\bigimage{images/moodboard.png}

%06
\begin{frame}
    \frametitle{Every saga has a beginning!}
    \vfill
    \begin{center}
    \large \textbf{Durante um teste de rotina, a nave Genesis é trasportada
    através de um \textit{wormhole} para o quadrante \textit{gamma} da galáxia,
    e precisa sobreviver à \textit{Guerra do Infinito}.}
    \vfill
    \large\textbf{O que era só um dia de testes virou\\uma luta pela sobrevivência.}
    \end{center}
    \vfill
\end{frame}

%06.b
\begin{frame}
    \frametitle{Na vida nada se cria...}

    \begin{itemize}
        \item Gradius/Nemesis
        \item Farscape
        \item Star Trek: Voyager
        \item Um filme muito, muito ruim...
    \end{itemize}
\end{frame}


%07
\begin{frame}[fragile]
    \frametitle{Modelo de Aplicação pygame}

    \begin{python}
        import pygame
        pygame.init()
        # inicia tela
        screen = pygame.display.set_mode((320,200))
        pygame.display.set_caption("Hello World!")
        # loop principal
        running = True
        while running:
            # trata eventos
            for event in pygame.event.get():
                if event.type == pygame.QUIT:
                    running = False
            # atualiza objetos
            # desenha objetos
            # pygame usa double buffer!
            pygame.display.update()
    \end{python}
\end{frame}

%08
\begin{frame}[fragile]
    \frametitle{O mínimo que voce precisa saber...}

    \begin{python}
    pygame.init()

    width, height = size = (800, 600)
    flags = pygame.FULLSCREEN | pygame.HWSURFACE | pygame.DOUBLEBUF
    screen = pygame.display.set_mode(size, flags)
    \end{python}
\end{frame}

%09
\begin{frame}[fragile]
    \frametitle{Desenhando na tela}
    \begin{python}
        python.draw.circle(screen, red, (100,100), 50)
        python.draw.polygon(screen, white, point_list)
        python.draw.rect(screen, white, (x, y, rect_w, rect_h))
        screen.blit(image, (x, y))
    \end{python}
\end{frame}

%10
\begin{frame}[fragile]
    \frametitle{Um universo a estrelar...}

    \begin{python}
    from random import randint, choice

    def create_star(x):
        y = randint(0, height)
        speed = choice([1, 2, 3])
        magnitude = choice([1, 2, 3])
        color = (coice(100, 200, 250),) * 3
        return (x, y, speed, magnitude, color)
    \end{python}
\end{frame}

%10.a
\begin{frame}[fragile]
    \frametitle{List Comprehension}

    \begin{itemize}
        \item É uma construção de Python que permite o processamento de
        uma lista de elementos.
    \end{itemize}

    \begin{python}
    starfield = [create_star(randint(0, width))
                 for star in range(count)]
    \end{python}
\end{frame}

%10.b
\begin{frame}[fragile]
    \frametitle{Operador Ternário}

    \begin{itemize}
        \item O operador ternário seleciona um entre dois valores, dada uma
        condição.
        \item Vai muito bem com \textit{guacamole} e \textit{list comprehension}.
    \end{itemize}

    \begin{python}
    stars = [(x - speed, y, speed, mag, color)
             if x - speed > 0
             else create_star(width)}
             for x, y, speed, mag, color in stars]
    \end{python}
\end{frame}

\begin{frame}[fragile]
    \frametitle{Desenhando a lista de estrelas}

    \begin{itemize}
        \item Os \textit{list comprehension} estão entre as estruturas mais
        eficientes para processar listas.
    \end{itemize}

    \begin{python}
        [python.draw.circle(screen, color, (x, y), mag)
         for x, y, _, mag, color in stars]
    \end{python}
\end{frame}

%10.d
\bigimage{images/starfield.png}

%11
\begin{frame}
    \frametitle{Sprites}

    \begin{itemize}
        \item Sprites são imagens 2D, mas nos jogos, eles tem movimento.
        \item O uso de sprites facilita a definição dos objetos móveis
        do jogo.
        \item Sprites, normalmente, tem suporte a transparência.
    \end{itemize}
\end{frame}

%11.a
\bigimage{images/f18-big.png}

%12
\begin{frame}
    \frametitle{Sprites com animação}

    \begin{itemize}
        \item pygame suporta imagens GIF, mas sem animação.
        \item É possível utilizar eventos para animar \textit{sprites}.
        \item pygame oferece diversos plugins que podem ser utilizados.
        \item Obviamente, existe um plugin para GIF animado (que não funciona).
    \end{itemize}
\end{frame}

%12.a
\begin{frame}
    \begin{columns}
        \begin{column}{0.3\textwidth}
            \includegraphics[width=3cm]{images/out0000.png}
            \vfill
            \includegraphics[width=3cm]{images/out0003.png}
            \vfill
            \includegraphics[width=3cm]{images/out0006.png}
            \vfill
            \includegraphics[width=3cm]{images/out0009.png}
        \end{column}
        \begin{column}{0.3\textwidth}
            \includegraphics[width=3cm]{images/out0001.png}
            \vfill
            \includegraphics[width=3cm]{images/out0004.png}
            \vfill
            \includegraphics[width=3cm]{images/out0007.png}
            \vfill
            \includegraphics[width=3cm]{images/out0010.png}
        \end{column}
        \begin{column}{0.3\textwidth}
            \includegraphics[width=3cm]{images/out0002.png}
            \vfill
            \includegraphics[width=3cm]{images/out0005.png}
            \vfill
            \includegraphics[width=3cm]{images/out0008.png}
            \vfill
            \includegraphics[width=3cm]{images/out0011.png}
        \end{column}
    \end{columns}
\end{frame}

%13
\begin{frame}
    \frametitle{Tratamento de Eventos}

    \begin{itemize}
        \item pygame oferece um sistema de eventos por \textit{polling}.
        \item Para criar um \textit{engine} com um \textit{loop} genérico,
        é preciso permitir que o código cliente seja chamado de volta.
        \item Um mecanismo desses permite que funções cliente sejam chamadas
        para eventos do pygame.
        \item \textit{E para felicidade geral da nação...\\ funções são} cidadãs de primeira ordem\textit{!}
    \end{itemize}
\end{frame}

%13.b
\begin{frame}[fragile]
    \frametitle{O loop de eventos}

    \begin{python}
    # loop genérico, em Game.run()
    while self.running:
        # handle events
        for event in pygame.event.get():
            handle_event(event)
    \end{python}
\end{frame}

\begin{frame}[fragile]
    \frametitle{Respondendo a eventos de teclado}

    \begin{python}
    def handle_event(event):
        if event.type in [pygame.KEYDOWN]:
            keydown(event)

    def keydown(event):
        for handler in keydown_handlers:
            handler(event)
    \end{python}

\end{frame}

%13.c
\begin{frame}[fragile]
    \frametitle{Respondendo a eventos de teclado}

    \begin{python}
# my code
def move(event):
    """Move player with directional keys."""
    keys = pygame.key.get_pressed()
    dx, dy = 0, 0
    dy = -1 if keys[pygame.K_UP] else 0
    dy = dy + 1 if keys[pygame.K_DOWN] else dy
    dx = -1 if keys[pygame.K_LEFT] else 0
    dx = dx + 1 if keys[pygame.K_RIGHT] else dx
    player.move = (dx * config.speed, dy * config.speed)

# Configuring the game object
game.on_keydown((pygame.K_UP, pygame.K_DOWN,
                pygame.K_LEFT, pygame.K_RIGHT), move)
game.on_keyup((pygame.K_UP, pygame.K_DOWN,
               pygame.K_LEFT, pygame.K_RIGHT), move)
    \end{python}
\end{frame}

%14
\bigtitle{E o que mais \\ falta fazer?}

%14.a
\begin{frame}
    \frametitle{Um jogo tem tanta coisa...}

    \begin{itemize}
        \item Tratamento de colisões.
        \item Comportamento de NPCs.
        \item Trocas de fazes.
        \item Cenários.
    \end{itemize}
\end{frame}

% 15
\bigtitle{Sem audio?}

%15.b
\begin{frame}
    \frametitle{pygame Mixer!}

    \begin{itemize}
        \item pygame tem um mixer que, sem configuração, suporta 8 canais de audio.
        \item Suporte a loops de áudio já embutido.
        \item Suporte a diversos formatos de áudio.
        \item Ogg Vorbis é a melhor opção.
        \item E a internet está cheia de loops \textit{royalty free}...
    \end{itemize}
\end{frame}

% 17
\bigtitle{E agora? \\ Pra onde vou?}

% 17.a
\bigtitle{\url{https://python.org} \vfill \url{https://pygame.org}}

%18
\begin{frame}
    \frametitle{Próximos Passos}

    \begin{itemize}
        \item Desenvolver um engine para criação de jogos 2D!
        \item Para ensinar programação orientada a objetos com Python.
        \item Para ensinar \textit{design} de jogos.
        \item Para ensinar criação de roteiros de jogos.
    \end{itemize}
\end{frame}

%18.a
\bigtitle{\Large E para criar jogos, né?}

%18.b
\bigtitle{\large Na verdade... \\ Esse nunca foi o objetivo...}

%19
\bigtitle{\Large Mas nem uma demonstração?}

%19.b
\bigtitle{\large \url{https://github.com/rafasgj/genesis}}

%20
\finalframe[Muito Obrigado!]{
    \url{mailto:rafasgj@gmail.com}\\
    \url{https://rafaeljeffman.com/tchelinux}
}

\end{document}
