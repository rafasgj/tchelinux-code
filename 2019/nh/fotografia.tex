\documentclass[aspectratio=169,14pt,usenames,dvipsnames]{beamer}

\usepackage[utf8]{inputenc}
\usepackage{enumitem}
\usepackage{calc}

\usepackage{datetime}
\newcommand\builddate{%
   \ifcase \month%
        \or Janeiro%
        \or Fevereiro%
        \or Março%
        \or Abril%
        \or Maio%
        \or Junho%
        \or Julho%
        \or Agosto%
        \or Setembro%
        \or Outubro%
        \or Novembro%
        \or Dezembro%
    \fi\space\number\year%
}

\newcommand{\loadtheme}[1]{%
    \input{themes/#1}%
}
\newcommand{\presentationlanguage}[1]{%
    \usepackage[#1]{babel}%
}

\newcommand{\usecodingsamples}[1]{%
    \usepackage{listings}%
    \input{listings/#1}%
}

% Configura a apresentação para ser executada em tela cheia.
\newcommand{\setfullscreen}{\hypersetup{pdfpagemode=FullScreen}}

% Hide beamer navigation simbols
\beamertemplatenavigationsymbolsempty

%
% Standard frames
%

% coverframe
\newcommand{\coverframe}{%
    \begin{frame} %
        \titlepage %
    \end{frame} %
}

% finalframe{email}
\newcommand{\finalframe}[2][Thank you!]{%
    \begin{frame}%
        \begin{flushright}%
            \huge \textbf{#1}%
            \vfill%
            \large \textbf{#2}%
        \end{flushright}%
    \end{frame}%
}

\newcommand{\bigtitle}[1]{%
    \begin{frame}%
        \begin{center}%
            \Huge {#1}%
        \end{center}%
    \end{frame}%
}



\loadtheme{photographie}

\title{Essa câmera faz fotos muito boas, né?}
\subtitle{}
\author{Rafael Guterres Jeffman}
\institute{}
\date{2019}

\begin{document}

%01
\coverframe

%02
\begin{frame}
    \frametitle{Fotografia}
    \begin{itemize}
        \item A arte de gravar a luz!
        \item Processos alternativos (cionotipia, platinum).
        \item Processos tradicionais (filme, dye print, E-6, C-41).
        \item Processos digitais.
    \end{itemize}
\end{frame}

\transition{1.0}{images/012.jpg}

%03
\begin{frame}
    \frametitle{Processo Fotográfico}
    \begin{columns}
        \begin{column}{0.4\textwidth}
            \begin{itemize}
                \item Uma ideia.
                \item As diversas capturas.
                \item A seleção (edição).
            \end{itemize}
        \end{column}
        \begin{column}{0.5\textwidth}
            \begin{itemize}
                \item Alterações na imagem.
                \item Publicação.
                \item Conservação.
            \end{itemize}
        \end{column}
    \end{columns}
\end{frame}

\transition{1.0}{images/004.jpg}

%04
\begin{frame}
    \frametitle{Fotografia Digital}
    \begin{itemize}
        \item É uma coisa muito parecida com fotografia.
        \item Durante um período do Flickr, eram feitas mais fotografias por
        dia do que em todo o primeiro século de história da fotografia.
        \vfill
        \item Envolve muitas e muitas linhas de código...
    \end{itemize}
\end{frame}

%05
\citation{Meu sobrinho faz mais barato.}{Um chato.}

%06
\begin{frame}
    \frametitle{Captura}
    \begin{itemize}
        \item Controle da câmera.
        \item Abertura, velocidade, sensibilidade...
        \item Atuação do disparador.
    \end{itemize}
\end{frame}

%07
\bigtitle{Gphoto2}

%08
\begin{frame}
    \frametitle{Seleção e Edição}
    \begin{itemize}
        \item Vizualização
        \item Tags
        \item Ratings
        \item Flagging
    \end{itemize}
\end{frame}

\transition{1.0}{images/002.jpg}

%09
\begin{frame}
    \frametitle{Softwares}
    \begin{itemize}
        \item gThumb
        \item Shotwell
        \item Geeqie
        \item Pix
        \item Digikam
    \end{itemize}
\end{frame}

%10
\citation{Imagina a exposição que esse trabalho vai te dar.}{Este espaço foi
deixado em branco por motivos de: horário da palestra.}

%11
\begin{frame}
    \frametitle{Manipulação Digital}
    \begin{itemize}
        \item Sim, ela é necessária.
        \item Não, não tem nada de errado com isso.
        \vfill
        \item "Ah... mas eu sou purista."
        \begin{itemize}
            \item Então usa cianotipia.
        \end{itemize}
    \end{itemize}
\end{frame}

\transition{1.0}{images/013.jpg}

%11.b
\begin{frame}
    \frametitle{Manipulação Digital}
    \begin{itemize}
        \item Ajustes de tom.
        \item Curvas e níveis.
        \item Crop, nivelamento e transformações.
        \item Pintura digital.
        \item Copiar, colar, apagar, ajustar.
    \end{itemize}
\end{frame}

%11.c
\bigtitle{Eu sei trocar cabeças. \\ {\Large ... e já precisei fazer ...}}

%12
\begin{frame}
    \frametitle{Software para Manipulação}
    \begin{itemize}
        \item Gimp
        \item Krita
        \item Hugin
        \item LuminanceHDR
        \item HDRMerge
    \end{itemize}
\end{frame}

\transition{1.0}{images/011.jpg}

%13
\begin{frame}
    \frametitle{Conversão RAW}
    \begin{itemize}
        \item Arquivos RAW trazem o máximo de informação da captura.
        \item Precisam ser convertidos para serem vistos.
        \item Não se edita um arquivo RAW, mas se influencia na conversão.
    \end{itemize}
\end{frame}

\transition{1.0}{images/010.jpg}

%14
\begin{frame}
    \frametitle{Conversores RAW}
    \begin{itemize}
        \item DCRAW
        \item UFRaw
        \item LightZone
        \item RAWTherapee
        \item darktable
    \end{itemize}
\end{frame}

%15
\citation{Livre não é sem autor!}{Vale para software e fotografia...}

%16
\begin{frame}
    \frametitle{E o que mais é possível?}
    \begin{itemize}
        \item Gerenciamento de Cores
        \begin{itemize}
            \item DispCAL
            \item Argill CMS
            \item DCamProf - Camera Profiler
        \end{itemize}
        \item Magic Lantern - Firmware para Canon.
    \end{itemize}
\end{frame}

\transition{1.0}{images/009.jpg}

%17
\begin{frame}
    \frametitle{E tem mais?}
    \begin{itemize}
        \item Rapid Photo Downloader
        \item Exiv2 e exiftool
        \item Forensically
        \item Siril
    \end{itemize}
\end{frame}

\transition{1.0}{images/006.jpg}

%18
\begin{frame}
    \frametitle{É muita coisa...}
    Então comece com...
    \begin{itemize}
        \item Digikam
        \item darktable
        \item Gimp
    \end{itemize}
\end{frame}

\transition{1.0}{images/008.jpg}

%19
\begin{frame}
    \frametitle{E os desastres?}
    Quem tem um, não tem nenhum...
    \begin{itemize}
        \item rsync
        \item Bacula
    \end{itemize}
\end{frame}

\transition{1.0}{images/005.jpg}

%20
\begin{frame}
    \frametitle{Mas tu nem falou de impressão!}
    \begin{itemize}
        \item CUPS
    \end{itemize}
    Mas aqui... a gente sofre um pouco mais...
\end{frame}

\transition{1.0}{images/001.jpg}

%21
\bigtitle{{\color{black} Começamos o bate-papo...}}

%22
\bigtitle{É possivel usar para as minhas fotos?}

%23
\bigtitle{Posso utilizar profissionalmente?}

%24
\bigtitle{Posso parar de pagar a Adobe?}

%25
\bigtitle{E vem novidade por aí... \\muitas novidades...}

%26
\finalframe[Muito Obrigado!]{
    \url{mailto:rafasgj@gmail.com}\\
    \url{https://rafaeljeffman.com/tchelinux}
}

%27
\bigimage{images/007.jpg}

\end{document}
